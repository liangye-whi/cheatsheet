\documentclass[10pt,landscape,fleqn]{ctexart}
\usepackage{multicol}
\usepackage{calc}
\usepackage{ifthen}
\usepackage[landscape]{geometry}
\usepackage{amsmath,amsthm,amsfonts,amssymb}
\usepackage{color,graphicx,overpic}
\usepackage{hyperref}
\usepackage{bm}
\newcommand{\brm}[1]{\bm{\mathrm{#1}}}
\newcommand{\bA}{\brm{A}}
\newcommand{\bi}{\brm{i}}
\newcommand{\bj}{\brm{j}}
\newcommand{\bk}{\brm{k}}
\newcommand{\be}{\brm{e}}
\newcommand{\pd}{\partial}
\setlength{\mathindent}{10pt}

\hypersetup{pdfinfo={
  /Title (Vector_Analysis.pdf)
  /Creator (TeX)
  /Producer (XeLaTeX 1.40.0)
  /Author (Ye)
  /Subject (Vector Analysis)
  /Keywords (vector, analysis, derivatives, gradient)}}

% This sets page margins to .5 inch if using letter paper, and to 1cm
% if using A4 paper. (This probably isn't strictly necessary.)
% If using another size paper, use default 1cm margins.
\ifthenelse{\lengthtest { \paperwidth = 11in}}
    { \geometry{top=.5in,left=.5in,right=.5in,bottom=.5in} }
    {\ifthenelse{ \lengthtest{ \paperwidth = 297mm}}
        {\geometry{top=1cm,left=1cm,right=1cm,bottom=1cm} }
        {\geometry{top=1cm,left=1cm,right=1cm,bottom=1cm} }
    }

% Turn off header and footer
\pagestyle{empty}

% Redefine section commands to use less space
\makeatletter
\renewcommand{\section}{\@startsection{section}{1}{0mm}%
                                {-1ex plus -.5ex minus -.2ex}%
                                {0.5ex plus .2ex}%x
                                {\normalfont\large\bfseries}}
\renewcommand{\subsection}{\@startsection{subsection}{2}{0mm}%
                                {-1explus -.5ex minus -.2ex}%
                                {0.5ex plus .2ex}%
                                {\normalfont\normalsize\bfseries}}
\renewcommand{\subsubsection}{\@startsection{subsubsection}{3}{0mm}%
                                {-1ex plus -.5ex minus -.2ex}%
                                {1ex plus .2ex}%
                                {\normalfont\small\bfseries}}
\renewcommand\tagform@[1]{}
\makeatother

% Define BibTeX command
\def\BibTeX{{\rm B\kern-.05em{\sc i\kern-.025em b}\kern-.08em
    T\kern-.1667em\lower.7ex\hbox{E}\kern-.125emX}}

% Don't print section numbers
\setcounter{secnumdepth}{0}


\setlength{\parindent}{0pt}
\setlength{\parskip}{0pt plus 0.5ex}

%My Environments
\newtheorem{example}[section]{Example}
% -----------------------------------------------------------------------

\begin{document}
\raggedright
\footnotesize
\begin{multicols}{3}


% multicol parameters
% These lengths are set only within the two main columns
%\setlength{\columnseprule}{0.25pt}
\setlength{\premulticols}{1pt}
\setlength{\postmulticols}{1pt}
\setlength{\multicolsep}{1pt}
\setlength{\columnsep}{2pt}

\begin{center}
    \Large{\underline{矢量分析与场论}} \footnotesize{\today}
\end{center}

\section{原始定义}
梯度:标量场某一点处增长率最大的方向(向量)\\
散度:向量场某一点处通量体密度(标量)\\
旋度:标量场某一点处最大的环量面密度(向量)

\section{矢量场的Jacobi矩阵}
设$\brm{A}=P\bi+Q\bj+R\bk$,其Jacobi矩阵为
\begin{equation}
    D\bA=\dfrac{\pd(P,Q,R)}{\pd(x,y,z)}=\begin{pmatrix}
    \dfrac{\pd P}{\pd x}&\dfrac{\pd P}{\pd y}&\dfrac{\pd P}{\pd z}\\
    \dfrac{\pd Q}{\pd x}&\dfrac{\pd Q}{\pd y}&\dfrac{\pd Q}{\pd z}\\
    \dfrac{\pd R}{\pd x}&\dfrac{\pd R}{\pd y}&\dfrac{\pd R}{\pd z}
\end{pmatrix}
\end{equation}
类似单变量函数的导数。对角元之和为散度,非对角元组合成旋度。

\section{Lam\'e系数与空间元}
从直角坐标系到任意正交曲线坐标系变换,已知$x,y,z$用新坐标变量的表达,可以求其Lam\'e系数:
\begin{equation}
    H_i(q_1,q_2,q_3)=\sqrt{\left(\dfrac{\pd x}{\pd q_i}\right)^2+\left(\dfrac{\pd y}{\pd q_i}\right)^2+\left(\frac{\pd z}{\pd q_i}\right)^2},\ i=1,2,3.
\end{equation}
或计算式,先算$x,y,z$用$q_1,q_2,q_3$表示的全微分:
\begin{equation}
    (dx)^2+(dy)^x+(dz)^2=H_1^2(dq_1)^2+H_2^2(dq_2)^2+H_3^2(dq_3)^2
\end{equation}
曲线弧微分为
\begin{equation}
    ds_i=H_idq_i
\end{equation}
面积元为
\begin{equation}
    dS_{ij}=H_iH_jdq_idq_j
\end{equation}
体积元为
\begin{equation}
    dV=H_1H_2H_3dq_1dq_2dq_3
\end{equation}
柱坐标
\begin{equation}
    H_\rho=1,\ H_\phi=\rho,\ H_z=1,\ dV=\rho d\rho d\phi dz
\end{equation}
球坐标
\begin{equation}
    H_r=1,\ H_\theta=r,\ H_\phi=r\sin\theta,\ dV=r^2\sin\theta drd\theta d\phi
\end{equation}

\section{基向量导数}
基向量与其导数正交。
\begin{align}
    \frac{\pd\be_1}{\pd q_1}=&-\frac{\be_2}{H_2}\frac{\pd H_1}{\pd q_2}-\frac{\be_3}{H_3}\frac{\pd H_1}{\pd q_3}\\
    \frac{\pd\be_1}{\pd q_2}=&\frac{\be_2}{H_1}\frac{\pd H_2}{\pd q_1}\hspace{3em}
    \frac{\pd\be_1}{\pd q_3}=\frac{\be_3}{H_1}\frac{\pd H_3}{\pd q_1}
\end{align}
柱坐标
\begin{equation}
    \dfrac{\pd \be_\rho}{\pd \phi} = \be_\phi\quad \dfrac{\pd \be_\phi}{\pd \phi}=-\be_\rho
\end{equation}
球坐标
\begin{align}
    &\dfrac{\pd \be_r}{\pd \theta} = \be_r\quad \dfrac{\pd \be_r}{\pd \phi}=\be_\phi\sin\theta\\
    &\dfrac{\pd \be_\theta}{\pd \theta}=-\be_r\quad\dfrac{\pd \be_\theta}{\pd \phi}=\be_\phi\cos\theta\quad \dfrac{\be_\phi}{\pd\phi}=-\be_r\sin\theta-\be_\theta r\cos\theta
\end{align}

\section{正交曲线坐标系$\nabla$}
\begin{equation}
    \nabla=\frac{\be_1}{H_1}\frac{\pd}{\pd q_1}+\frac{\be_2}{H_2}\frac{\pd}{\pd q_2}+\frac{\be_3}{H_3}\frac{\pd}{\pd q_3}
\end{equation}
先求导再求内积。

柱坐标
\begin{align}
    &\nabla u=\frac{\pd u}{\pd\rho}\be_\rho+\frac{1}{\rho}\frac{\pd u}{\pd \phi}\be_\phi+\frac{\pd u}{\pd z}\be_z\\
    &\nabla\cdot\bA=\frac{1}{\rho}\frac{\pd\rho A_\rho}{\pd \rho}+\frac{1}{\rho}\frac{\pd A_\phi}{\pd \phi}+\frac{\pd A_z}{\pd z}\\
    &\nabla\times\bA=\left[\frac{1}{\rho}\frac{\pd A_z}{\pd \phi}-\frac{\pd A_\phi}{\pd z}\right]\be_\rho+\left[\frac{\pd A_\rho}{\pd z}-\frac{\pd A_z}{\pd \rho}\right]\be_\phi\nonumber\\
    &\qquad+\frac{1}{\rho}\left[\frac{\pd \rho A_\phi}{\pd\rho}-\frac{\pd A_\rho}{\pd\phi}\right]\be_z\\
    &\nabla^2u=\frac{\pd^2u}{\pd\rho^2}+\frac{1}{\rho}\frac{\pd u}{\pd\rho}+\frac{1}{\rho^2}\frac{\pd^2 u}{\pd\phi^2}+\frac{\pd^2u}{\pd z^2}
\end{align}
球坐标
\begin{align}
    &\nabla u=\frac{\pd u}{\pd r}\be_r+\frac1r\frac{\pd u}{\pd\theta}\be_\theta+\frac1{r\sin\theta}\frac{\pd u}{\pd\phi}\be_\phi\\
    &\nabla\cdot\bA=\frac1{r^2}\frac{\pd r^2A_r}{\pd r}+\frac{1}{r\sin\theta}\frac{\pd \sin\theta A_\theta}{\pd\theta}+\frac1{r\sin\theta}\frac{\pd A_\phi}{\pd\phi}\\
    &\nabla\times\bA=\frac{1}{r\sin\theta}\left[\frac{\pd\sin\theta A_\phi}{\pd\theta}-\frac{\pd A_\theta}{\pd \phi}\right]\be_r+\frac1r\left[\frac1{\sin\theta}\frac{\pd A_r}{\pd\phi}-\frac{\pd rA_\phi}{\pd r}\right]\be_\theta\nonumber\\
    &\qquad+\frac1r\left[\frac{\pd rA_\theta}{\pd r}-\frac{\pd A_r}{\pd\theta}\right]\be_\phi\\
    &\nabla^2 u=\frac1{r^2}\frac{\pd}{\pd r}r^2\frac{\pd u}{\pd r}+\frac{1}{r^2\sin\theta}\dfrac{\pd}{\pd\theta}\left(\sin\theta\frac{\pd u}{\pd\theta}\right)+\frac{1}{r^2\sin^2\theta}\frac{\pd^2u}{\pd\phi^2}
\end{align}
% You can even have references
%\rule{0.3\linewidth}{0.25pt}
%\scriptsize
%\bibliographystyle{abstract}
%\bibliography{refFile}
\end{multicols}
\end{document}
