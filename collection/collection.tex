\documentclass[10pt,landscape,fleqn]{ctexart}
\usepackage{multicol}
\usepackage{calc}
\usepackage{ifthen}
\usepackage[landscape]{geometry}
\usepackage{amsmath,amsthm,amsfonts,amssymb}
\usepackage{color,graphicx,overpic}
\usepackage{hyperref}
\usepackage{bm}
\newcommand{\brm}[1]{\bm{\mathrm{#1}}}
\newcommand{\bA}{\brm{A}}
\newcommand{\bi}{\brm{i}}
\newcommand{\bj}{\brm{j}}
\newcommand{\bk}{\brm{k}}
\newcommand{\be}{\brm{e}}
\newcommand{\pd}{\partial}
\setlength{\mathindent}{10pt}

\hypersetup{pdfinfo={
  /Title (collection.pdf)
  /Creator (TeX)
  /Producer (XeLaTeX 1.40.0)
  /Author (Ye Liang)
  /Subject (CheatSheet)
  /Keywords (collection, cheatsheet)}}

% This sets page margins to .5 inch if using letter paper, and to 1cm
% if using A4 paper. (This probably isn't strictly necessary.)
% If using another size paper, use default 1cm margins.
\ifthenelse{\lengthtest { \paperwidth = 11in}}
{ \geometry{top=.5in,left=.5in,right=.5in,bottom=.5in} }
{\ifthenelse{ \lengthtest{ \paperwidth = 297mm}}
    {\geometry{top=1cm,left=1cm,right=1cm,bottom=1cm} }
    {\geometry{top=1cm,left=1cm,right=1cm,bottom=1cm} }
}


% Turn off header and footer
\pagestyle{empty}

% Redefine section commands to use less space
\makeatletter
\renewcommand{\section}{\@startsection{section}{1}{0mm}%
                                {-1ex plus -.5ex minus -.2ex}%
                                {0.5ex plus .2ex}%x
                                {\normalfont\large\bfseries}}
\renewcommand{\subsection}{\@startsection{subsection}{2}{0mm}%
                                {-1explus -.5ex minus -.2ex}%
                                {0.5ex plus .2ex}%
                                {\normalfont\normalsize\bfseries}}
\renewcommand{\subsubsection}{\@startsection{subsubsection}{3}{0mm}%
                                {-1ex plus -.5ex minus -.2ex}%
                                {1ex plus .2ex}%
                                {\normalfont\small\bfseries}}
% Don't print equation numbers
\renewcommand\tagform@[1]{}
\makeatother

% Define BibTeX command
\def\BibTeX{{\rm B\kern-.05em{\sc i\kern-.025em b}\kern-.08em
    T\kern-.1667em\lower.7ex\hbox{E}\kern-.125emX}}

% Don't print section numbers
\setcounter{secnumdepth}{0}

\setlength{\parindent}{0pt}
\setlength{\parskip}{0pt plus 0.5ex}

%My Environments
\newtheorem{example}[section]{Example}
% -----------------------------------------------------------------------

\begin{document}
\raggedright
\footnotesize
\begin{multicols}{3}

% multicol parameters
% These lengths are set only within the two main columns
%\setlength{\columnseprule}{0.25pt}
\setlength{\premulticols}{1pt}
\setlength{\postmulticols}{1pt}
\setlength{\multicolsep}{1pt}
\setlength{\columnsep}{2pt}

\begin{center}
    \Large{\underline{数学物理每日默写}} \footnotesize{13FEB2018}
\end{center}

\section{高斯函数积分}
基向量与其导数正交。
\begin{equation}
    \int_{-\infty}^\infty e^{-ax^2}dx=\sqrt{\dfrac{\pi}{a}}
\end{equation}

\section{$\delta$函数Fourier变换}
\begin{equation}
    \int_{-\infty}^\infty e^{ikx}dk=2\pi\delta(x)
\end{equation}

\section{梯度与调和量}
柱坐标
\begin{align}
    &\nabla=\be_\rho\frac{\pd}{\pd\rho}+\be_\phi\frac{1}{\rho}\frac{\pd}{\pd \phi}+\be_z\frac{\pd}{\pd z}\\
    &\nabla^2=\frac1{\rho}\frac{\pd}{\pd}\rho\frac{\pd}{\pd\rho}+\frac{1}{\rho^2}\frac{\pd^2 }{\pd\phi^2}+\frac{\pd^2}{\pd z^2}
\end{align}
球坐标
\begin{align}
    &\nabla=\be_r\frac{\pd}{\pd r}+\be_\theta\frac1r\frac{\pd }{\pd\theta}+\be_\phi\frac1{r\sin\theta}\frac{\pd }{\pd\phi}\\
    &\nabla^2 =\frac1{r^2}\frac{\pd}{\pd r}r^2\frac{\pd}{\pd r}+\frac{1}{r^2\sin\theta}\dfrac{\pd}{\pd\theta}\left(\sin\theta\frac{\pd }{\pd\theta}\right)+\frac{1}{r^2\sin^2\theta}\frac{\pd^2}{\pd\phi^2}
\end{align}

\section{格林恒等式}
第一恒等式
\begin{equation}
    \int_V(\psi\nabla^2\phi+\nabla\phi\cdot\nabla\psi)dV=\oint_S\psi\frac{\pd \phi}{\pd n}dS
\end{equation}
第二恒等式
\begin{equation}
    \int_V(\psi\nabla^2\phi-\phi\nabla^2\psi)dV=\oint_S\left(\psi\frac{\pd\phi}{\pd n}-\phi\frac{\pd\psi}{\pd n}\right)dS
\end{equation}

% You can even have references
%\rule{0.3\linewidth}{0.25pt}
%\scriptsize
%\bibliographystyle{abstract}
%\bibliography{refFile}
\end{multicols}
\end{document}
